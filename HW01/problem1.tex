\begin{questions}
\question{
Speed of light
}
\begin{solution}
To find the speed of light in atomic units it would be useful to remember the \textit{fine structure constant} defined as
\begin{equation}
  \frac{e^2}{4\pi \epsilon_0\hbar c} = \frac{1}{137},
  \label{fine:stru}
\end{equation}
but since $e=\hbar = 1/4\pi\epsilon_0 = 1$  eq. \ref{fine:stru} turns into
\begin{eqnarray}
  \frac{1}{c} = \frac{1}{137},\nonumber\\
  \Rightarrow \hlgreen{c = 137}.
\end{eqnarray}

\end{solution}

\question{Bohr's Magneton}
\begin{solution}
  Bohr magneton is defined as
  \begin{equation}
    \mu_B = \frac{e\hbar}{2m_e},
  \end{equation}
  now, from the fact that $e=\hbar = m_e=1$ we can deduce the numerical value
  \begin{equation}
    \mu_B = \frac{e\hbar}{2m_e} = \hlgreen{\frac{1}{2}}.
  \end{equation}

  In both questions we went from SI units to atomic ones.

\end{solution}

\end{questions}
